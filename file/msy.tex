%!TEX program = xelatex
%% start of file `template-zh.tex'.
%% Copyright 2006-2013 Xavier Danaux (xdanaux@gmail.com).
%
% This work may be distributed and/or modified under the
% conditions of the LaTeX Project Public License version 1.3c,
% available at http://www.latex-project.org/lppl/.


\documentclass[12pt,a4paper,sans]{moderncv}   % possible options include font size ('10pt', '11pt' and '12pt'), paper size ('a4paper', 'letterpaper', 'a5paper', 'legalpaper', 'executivepaper' and 'landscape') and font family ('sans' and 'roman')

% moderncv 主题
\moderncvstyle{classic}                        % 选项参数是 ‘casual’, ‘classic’, ‘oldstyle’ 和 ’banking’
\moderncvcolor{blue}                          % 选项参数是 ‘blue’ (默认)、‘orange’、‘green’、‘red’、‘purple’ 和 ‘grey’
%\nopagenumbers{}                             % 消除注释以取消自动页码生成功能

% 字符编码
% \usepackage[utf8]{inputenc}                   % 替换你正在使用的编码
% \usepackage{CJKutf8}


% 调整页面出血
\usepackage[scale=0.88]{geometry}
%\setlength{\hintscolumnwidth}{3cm}           % 如果你希望改变日期栏的宽度

\usepackage{fontspec}
\usepackage{xunicode}
\usepackage{xeCJK}
\setmainfont{Minion Pro}
\setsansfont{Myriad Pro}
\setmonofont{Courier New}
\setCJKmainfont{方正准圆简体}
\setCJKsansfont{KaiTi}
\setCJKmonofont{SimHei}
%\setCJKmathfont{}

% 个人信息
\name{马}{申彦}
\title{web前端开发}                     % 可选项、如不需要可删除本行
\address{中国地质大学(武汉)}                          % 可选项、如不需要可删除本行
\phone[mobile]{187~7105~8712}              % 可选项、如不需要可删除本行
% \phone[fixed]{+2~(345)~678~901}               % 可选项、如不需要可删除本行
% \phone[fax]{+3~(456)~789~012}                 % 可选项、如不需要可删除本行
\email{msyfls123@foxmail.com}                    % 可选项、如不需要可删除本行
\homepage{msyfls123.github.io}                  % 可选项、如不需要可删除本行
\extrainfo{江苏无锡人}                 % 可选项、如不需要可删除本行
\photo[64pt][0pt]{msy.jpg}                  % ‘64pt’是图片必须压缩至的高度、‘0.4pt‘是图片边框的宽度 (如不需要可调节至0pt)、’picture‘ 是图片文件的名字;可选项、如不需要可删除本行
\quote{
	\small {
	想了解更多? \\
	\url{http://msyfls123.github.io}
	}
}                          % 可选项、如不需要可删除本行

\newcommand{\high}{\textcolor[rgb]{0.2,0.5,0.8}}
\newcommand{\tips}{\textcolor[rgb]{0.35,0.35,0.35}}
% 显示索引号;仅用于在简历中使用了引言
%\makeatletter
%\renewcommand*{\bibliographyitemlabel}{\@biblabel{\arabic{enumiv}}}
%\makeatother

% 分类索引
%\usepackage{multibib}
%\newcites{book,misc}{{Books},{Others}}
%----------------------------------------------------------------------------------
%            内容
%----------------------------------------------------------------------------------
\begin{document}
% \begin{CJK}{UTF8}{gbsn}                       % 详情参阅CJK文件包
\maketitle

\section{教育背景}
\cventry{2014 - 至今}{硕士}{中国地质大学}{武汉}{}{设计学}  % 第3到第6编码可留白
\cventry{2010 - 2014}{学士}{中国地质大学}{武汉}{}{工业设计}

\section{实践经历}
\cventry{2016.3-7}{前端开发(实习)}{\high{苏州蜗牛数字科技股份有限公司}}{}{}{
\begin{itemize}
	\item 编写具有适配性、响应用户交互操作的游戏活动专题页面:
	\tips{\begin{description}%
		\item[类型:]{预约+报名+抽奖+签到+手机投票+微信公众号回复关键词等}
		\item[开发:]{Gulp/webpack打包,Sass编写样式,小图片sprite化,Vue/React编写多交互页面}
		\item[问题:]{跨域Ajax,IE兼容及平稳退化,响应式,用户状态数据的维护等}
	\end{description}}
	\item 蜗牛直播平台(\url{v.woniu.com})的前端开发,解决问题:
	\tips{\begin{itemize}%
		\item 根据不同类型(flash/iframe/video)动态渲染直播源至页面
		\item 视频列表页面的js分页及按页面宽度缩放列表项以充满屏幕宽度
		\item 针对不同屏幕及不同内容长度对侧边栏、主体内容和底栏做兼容性适配
	\end{itemize}}
	\item 某海外视频站点的前端开发,解决问题:
	\tips{\begin{itemize}
		\item 首页及列表页的无限滚动加载
		\item Vue做用户评论点赞功能的组件化开发
		\item 未购买用户视频播放时的购买弹窗功能
	\end{itemize}}
\end{itemize}
}
\cventry{2014.7-8}{前端开发(实习)}{\high{武汉七彩马科技有限公司}}{}{}{
\begin{itemize}
	\item 静态展示页面的设计与制作
	\item jQuery完成悬浮导航,树状流动,页面锚点滚动及旋转标语展示等页面效果
\end{itemize}
}
% \cventry{2013.8}{生产部实习生}{无锡协友机械制造有限公司}{无锡}{}{
% 学习机械加工的流程及工艺特点,合作完成某设备塑胶卡件的加工
% }
\cventry{2013.7}{交流学生}{\high{韩国建国大学夏令营}}{}{}{
赴韩国建国大学参观学习,参与部分合作交流项目
}
\cventry{2013.5-6}{获奖}{\high{北京设计周}}{}{}{
获得2013年诺基亚·绿色设计大赛银奖,作品“环保垃圾桶”在设计周上展出
}

\section{个人技能}
\cvitem{前端开发}{\small{熟练开发符合W3C、兼容性及响应式要求的页面
		\begin{description}
			\item[\high{用框架}]{熟练使用jQuery/Vue/React开发应用,了解Angular,正在学习Redux等状态管理工具}
			\item[\high{工程化}]{应用Gulp/webpack等打包工具管理项目依赖的js/scss/img等资源,实现单页面应用}
			\item[\high{黑魔法}]{全景播放器Krpano,会它的一点点xml语法,能实现简单的交互}
		\end{description}}}
\cvitem{基础技能}{用	Git/SVN进行版本管理,npm做node包管理,Linux下简单操作,Vim会保存退出...}
\cvitem{后端开发}{\small{会Python/NodeJS,熟悉Django框架,能配置Nginx/MySQL}}
\cvitem{数据可视化}{\small{熟悉D3.js/HightCharts等框架,了解H5的Canvas/SVG语法,会用\LaTeX{}进行排版}}
\cvitem{平面设计}{\small{熟练掌握AI/Ps/Id等常用平面软件操作,排排小册子没问题的啦}}
\cvitem{交互设计}{\small{能使用Axure/AE/PR等软件完成APP网页等原型及交互,会制作简单小视频}}
\cvitem{三维设计}{\small{掌握Rhino/3dMax/SketchUp/CATIA等三维软件,有过虚拟仪器开发经验}}

\section{项目经历}
\cvitemwithcomment{2016}{
	深圳市中科德睿智能科技有限公司网站\qquad\url{derui-tech.com}}{ALL}
\cvitemwithcomment{2015}{中国地质大学(武汉)\qquad\url{jidian.cug.edu.cn}}{交互及前端开发}
\cvitemwithcomment{2014}{VDP.ICIDO软件在工业设计教学中的应用}{虚拟仿真软件应用}
\cvitemwithcomment{2013}{某免费停车概念APP原型}{原型设计制作}
\cvitemwithcomment{2013-2014}{中地大资源学院油藏物理虚拟实验室}{三维造型及贴图设计}
\cvitemwithcomment{2013-2014}{宜巴高速公路安全施工标准化图册}{三维造型及排版}

\section{获奖证书}
\cvitemwithcomment{CET6}{大学英语六级考试}{505}
\cvitemwithcomment{NCRE}{全国计算机等级考试}{二级(C语言)}
\cvitemwithcomment{比赛}{2013年诺基亚·绿色设计大赛 暨 龙腾奖青年设计之星}{银奖}
\cvitemwithcomment{比赛}{2014年CUIDC·全国大学生工业设计大赛}{入围奖}
\cvitemwithcomment{学校}{奖学金}{2012校长奖学金/2014杰美特奖学金}
\cvitemwithcomment{学校}{荣誉称号}{2012校优秀学生干部/2013校优秀共青团员}


\renewcommand{\listitemsymbol}{-}             % 改变列表符号

\section{自我评价}
\cvitem{爱好}{\small{电影、做饭、阅读、PC游戏}}
\cvitem{专注于}{\small{一切有趣的、有价值意义的事情}}

% 来自BibTeX文件但不使用multibib包的出版物
%\renewcommand*{\bibliographyitemlabel}{\@biblabel{\arabic{enumiv}}}% BibTeX的数字标签
% \nocite{*}
% \bibliographystyle{plain}
% \bibliography{publications}                    % 'publications' 是BibTeX文件的文件名

% 来自BibTeX文件并使用multibib包的出版物
%\section{出版物}
%\nocitebook{book1,book2}
%\bibliographystylebook{plain}
%\bibliographybook{publications}               % 'publications' 是BibTeX文件的文件名
%\nocitemisc{misc1,misc2,misc3}
%\bibliographystylemisc{plain}
%\bibliographymisc{publications}               % 'publications' 是BibTeX文件的文件名

% \clearpage\end{CJK}
\end{document}


%% 文件结尾 `template-zh.tex'.
