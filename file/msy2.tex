%!TEX program = xelatex
%% start of file `template-zh.tex'.
%% Copyright 2006-2013 Xavier Danaux (xdanaux@gmail.com).
%
% This work may be distributed and/or modified under the
% conditions of the LaTeX Project Public License version 1.3c,
% available at http://www.latex-project.org/lppl/.


\documentclass[10pt,a4paper,sans]{moderncv}   % possible options include font size ('10pt', '11pt' and '12pt'), paper size ('a4paper', 'letterpaper', 'a5paper', 'legalpaper', 'executivepaper' and 'landscape') and font family ('sans' and 'roman')

% moderncv 主题
\moderncvstyle{classic}                        % 选项参数是 ‘casual’, ‘classic’, ‘oldstyle’ 和 ’banking’
\moderncvcolor{blue}                          % 选项参数是 ‘blue’ (默认)、‘orange’、‘green’、‘red’、‘purple’ 和 ‘grey’
%\nopagenumbers{}                             % 消除注释以取消自动页码生成功能

% 字符编码
% \usepackage[utf8]{inputenc}                   % 替换你正在使用的编码
% \usepackage{CJKutf8}


% 调整页面出血
\usepackage[scale=0.92]{geometry}
%\setlength{\hintscolumnwidth}{3cm}           % 如果你希望改变日期栏的宽度

\usepackage{fontspec}
\usepackage{xunicode}
\usepackage{xeCJK}
\setmainfont{Minion Pro}
\setsansfont{Myriad Pro}
\setmonofont{Courier New}
\setCJKmainfont{微软雅黑}
\setCJKsansfont{KaiTi}
\setCJKmonofont{SimHei}
%\setCJKmathfont{}

% 个人信息
\name{马}{申彦}
\title{前端/交互/数据可视化}                     % 可选项、如不需要可删除本行
\address{中国地质大学(武汉)}                          % 可选项、如不需要可删除本行
\phone[mobile]{+86~187~7105~8712}              % 可选项、如不需要可删除本行
% \phone[fixed]{+2~(345)~678~901}               % 可选项、如不需要可删除本行
% \phone[fax]{+3~(456)~789~012}                 % 可选项、如不需要可删除本行
\email{745784917@qq.com}                    % 可选项、如不需要可删除本行
\homepage{github.com/msyfls123}                  % 可选项、如不需要可删除本行
\extrainfo{籍贯:江苏无锡}                 % 可选项、如不需要可删除本行
\photo[64pt][0.4pt]{msy.jpg}                  % ‘64pt’是图片必须压缩至的高度、‘0.4pt‘是图片边框的宽度 (如不需要可调节至0pt)、’picture‘ 是图片文件的名字;可选项、如不需要可删除本行
\quote{
	\small {
	想了解更多? - 
	\url{http://msyfls123.github.io}
	\\ 部分作品展示 - 
	\url{http://www.douban.com/photos/album/155070126/}
	}
}                          % 可选项、如不需要可删除本行

% 显示索引号;仅用于在简历中使用了引言
%\makeatletter
%\renewcommand*{\bibliographyitemlabel}{\@biblabel{\arabic{enumiv}}}
%\makeatother

% 分类索引
%\usepackage{multibib}
%\newcites{book,misc}{{Books},{Others}}
%----------------------------------------------------------------------------------
%            内容
%----------------------------------------------------------------------------------
\begin{document}
% \begin{CJK}{UTF8}{gbsn}                       % 详情参阅CJK文件包
\maketitle

\section{教育背景}
\cventry{2014 - 至今}{硕士}{中国地质大学}{武汉}{\textit{4.13}}{设计学}  % 第3到第6编码可留白
\cventry{2010 - 2014}{学士}{中国地质大学}{武汉}{\textit{3.62}}{工业设计}

\section{学习经历}
\cventry{2015.7-10}{设计实习生}{佛山尚致设计有限公司}{佛山}{\url{www.bob-id.com.cn}}{
参与多项平面设计及交互设计工作,利用网络推动公司形象宣传,完成了公司首个众筹产品的发布
}
\cventry{2014.7-8}{前端开发}{武汉七彩马科技有限公司}{武汉}{\url{7caima.com/static/service}}{
\begin{itemize}%
\item 参与静态展示页面的设计与制作
\item 独立完成悬浮导航,树状流动,页面锚点滚动及旋转标语展示等JS特效
\end{itemize}
}
% \cventry{2013.8}{生产部实习生}{无锡协友机械制造有限公司}{无锡}{}{
% 学习机械加工的流程及工艺特点,合作完成某设备塑胶卡件的加工
% }
\cventry{2013.7}{交流学生}{韩国建国大学夏令营}{韩国忠州}{}{
赴韩国参观学习两周,了解当地风土人情及先进设计技术
}
\cventry{2013.5-6}{获奖}{北京设计周}{北京}{}{
获得2013年诺基亚·绿色设计大赛银奖,作品“环保垃圾桶”在设计周上展出
}

\section{项目经历}
\cvitemwithcomment{2015}{中国地质大学(武汉)机电学院网站}{网页交互及前端开发}
\cvitemwithcomment{2014}{VDP.ICIDO软件在工业设计教学中的应用}{虚拟仿真软件应用}
\cvitemwithcomment{2013}{某免费停车概念APP原型}{原型设计制作}
\cvitemwithcomment{2013-2014}{中地大资源学院油藏物理虚拟实验室}{三维造型及贴图设计}
\cvitemwithcomment{2013-2014}{宜巴高速公路安全施工标准化图册}{三维造型及排版}

\section{获奖证书}
\cvitemwithcomment{CET6}{大学英语六级考试}{505}
\cvitemwithcomment{NCRE}{全国计算机等级考试}{二级(C语言)}
\cvitemwithcomment{比赛}{2013年诺基亚·绿色设计大赛 暨 龙腾奖青年设计之星}{银奖}
\cvitemwithcomment{比赛}{2014年CUIDC·全国大学生工业设计大赛}{入围奖}
\cvitemwithcomment{学校}{奖学金}{2012校长奖学金/2014杰美特奖学金}
\cvitemwithcomment{学校}{荣誉称号}{2012校优秀学生干部/2013校优秀共青团员}

\section{个人技能}
\cvitem{前端开发}{\small{熟练应用HTML/CSS/Javascript编写页面,掌握jQuery/AngularJS等前端框架}}
\cvitem{后端开发}{\small{熟悉Django/Python框架,能够部署Nginx/MySQL服务器}}
\cvitem{数据可视化}{\small{熟悉D3.js/HightCharts/Processing等语言框架,了解H5的Canvas/SVG语法,会用\LaTeX{}进行排版}}
\cvitem{平面设计}{\small{熟练掌握AI/Ps/Id等常用平面软件操作,能独立进行VI画册等设计}}
\cvitem{交互设计}{\small{能使用Axure/AE/PR等软件完成APP网页等交互效果,会制作简单小视频}}
\cvitem{三维设计}{\small{掌握Rhino/3dMax/SketchUp/CATIA等三维软件,有过虚拟仪器开发经验}}

\renewcommand{\listitemsymbol}{-}             % 改变列表符号

\section{自我评价}
\cvitem{爱好}{\small{电影、做饭、阅读、PC游戏}}
\cvitem{专注于}{\small{一切有趣的、有价值意义的事情}}

% 来自BibTeX文件但不使用multibib包的出版物
%\renewcommand*{\bibliographyitemlabel}{\@biblabel{\arabic{enumiv}}}% BibTeX的数字标签
% \nocite{*}
% \bibliographystyle{plain}
% \bibliography{publications}                    % 'publications' 是BibTeX文件的文件名

% 来自BibTeX文件并使用multibib包的出版物
%\section{出版物}
%\nocitebook{book1,book2}
%\bibliographystylebook{plain}
%\bibliographybook{publications}               % 'publications' 是BibTeX文件的文件名
%\nocitemisc{misc1,misc2,misc3}
%\bibliographystylemisc{plain}
%\bibliographymisc{publications}               % 'publications' 是BibTeX文件的文件名

% \clearpage\end{CJK}
\end{document}


%% 文件结尾 `template-zh.tex'.
